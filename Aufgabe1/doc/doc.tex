\documentclass{scrreprt}

\usepackage[utf8]{inputenc}
\usepackage[T1]{fontenc}
\usepackage {ngerman}
\usepackage{fancyvrb}
\usepackage{xcolor}
\usepackage{textcomp}

\author {Team "`Herder"' -- Robert Clausecker}
\title {L"osung zur Aufgabe 1}
\subtitle {des 29. Bundeswettbewerbs Informatik}

%Definitionen für den Formatierer
\include{definitions}

\begin{document}
\maketitle

\tableofcontents \newpage

\chapter{Bildbeschreibung}

\section{K"unstlerische Gedanken}
Die Grundidee der Zeichenanweisung war es, eine Art Bambus darzustellen, die der
traditionellen chinesischen Malerei in abstrakter Form "ahneln soll.  Dazu
sollen soweit m"oglich ausschlie"slich einfache geometrische Figuren verwendet
werden.  Aus "asthetischen (und zeitlichen) Gr"unden habe ich in diesem Beispiel
auf Farbe verzichtet.

Aufgrund meiner mangelnden Erfahrung mit dem Programm CFDG und der starken
Restriktion der Eingabedatei ist letztlich doch etwas anderes dabei
herausgekommen.  Trotzdem sehe ich meine (k"unstlerischen) Erwartungen in diesem
Bild erf"ullt.

Das Bild gleicht jetzt eher einem mehr oder weniger sch"onem Baum, der starke,
nahezu fraktale Ver"astelungen aufweist.  Je nach Zufallswert kommen sehr
unterschiedliche Formen zu Tage, mal ist der Baum sehr gestreckt, mal eher wie
ein Busch -- dies zeigt die gro"se Variation der Ausgabe, die CFDG aus einer
einzigen Regel erzeugen kann.

\section{Aufbau der Form}

Das Programm verwendet eine einzige Regel, sowie zwei "`von Hand"' hergestellte
Pfade, um Ausgabe zu erzeugen.  O.\,g. Regel \emph{bambus} kann drei m"ogliche
Formen erzeugen:

\begin{description}
\item[Stamm] Es wird ein langer Block erzeugt, an dessen Spitze mit schmalem
  Abstand (entsprechend der chinesischen Malerei) das n"achste Teilst"uck
  auf 90\,\% verkleinert angesetzt wird.  Der Stamm ist viermal so lang wie
  breit und wird aus dem Pfad \emph{stamm} erzeugt. Er erscheint als Form mit
  einer Wahrscheinlichkeit von 50\,\%. 
  
\item[Ast rechts] Diese Form erzeugt eine Verzweigung nach rechts.  Die Form
  besteht aus einem speziellen Pfad den ich \emph{T-St"uck} nenne,\footnote{Um
  keine Umlaute in den Bezeichnern zu verwenden, hei"st diese Form im Quelltext
  \emph{zweig}.} Der Pfad ist nach geometrischen Prinzipien konzipiert, die ich
  weiter unten erkl"are.  Er erzeugt einen auf 50\,\% skalierten Ast nach rechts
  und f"uhrt den Stamm mit der gleichen Skalierung wie auch beim Stamm (90\,\%)
  weiter.  Der Ast hat eine Drehung von c.\,a. 64\textdegree nach rechts, der
  Stamm wird um etwa 29\textdegree nach rechts gedreht.  Diese Form erscheint
  mit einer Wahrscheinlichkeit von 25\,\%.

\item[Ast links] Diese Form erzeugt einen Ast nach links und ist eine genaue
  Spiegelung des Astes nach rechts; sie entsteht auf die gleiche Art und Weise.
  Auch sie erscheint mit einer Wahrscheinlichkeit von 25\,\%.
\end{description}

\section{Das T-St"uck}

To be continued. % TODO Fertigmachen!!!

\chapter{Quelltext (bambus.cfdg)}

Weil das zum Formatieren des Quelltextes genutzte Skript \emph{pygmentize} keine
Unterst"utzung f"ur CFDG bietet und ich die Menge der Abh"angigkeiten nicht noch
weiter erh"ohen wollte, gibt es leider kein Highlighting f"ur diese Aufgabe.

Ich bitte, dies zu entschuldigen.
\input{bambus.cfdg.tex}

\chapter{Beispielausgaben}

To be continued. % TODO Fertigmachen!!!
\end{document}
