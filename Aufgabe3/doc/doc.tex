\documentclass{scrreprt}

\usepackage[utf8]{inputenc}
\usepackage[T1]{fontenc}
\usepackage {ngerman}
\usepackage{fancyvrb}
\usepackage{xcolor}

\author {Robert Clausecker -- Team "`Herder"'}
\title {L"osung zur Aufgabe 3}
\subtitle {des 29. Bundeswettbewerbs Informatik}

%Definitionen für den Formatierer
\include{definitions}

\begin{document}
\maketitle

\tableofcontents \newpage


\chapter {Teilaufgabe 1: Keine Optimierungen}

\section {Grundidee}
Die Idee zur L"osung dieser Teilaufgabe ist es, pro Lager nachzuschauen, wie
viele Fahrzeuge pro Tag an- bzw. abfahren.  Da ein Fahrzeug pro Tag genau eine
Strecke zur"ucklegen darf, kann man nun eine Art Fahrzeugbestand f"uhren.
Nachdem das Programm die Anzahl der an- und abfahrenden Fahrzeuge berechnet
hat, wird diese in den aktuellen Fahrzeugbestand einberechnet.

\section {Algorithmus}
Der Fahrzeugbestand wird als Paar nat"urlicher Zahlen dargestellt, die
erste stellt die Anzahl der sich an dem Lager befindlichen Fahrzeuge dar, die
zweite ist der Wert, wieviele Fahrzeuge insgesamt zu dieser Station geordert
werden m"ussen.  Aus dem Paar der anfahrenden- und abfahrenden Fahrzeuge sowie
dem vorherigen Bestand wird nach folgendem Schema der neue Bestand bestimmt:

Der Tupel der An- und abfahrenden Fahrzeuge sei $(an,ab)$, der
Tupel des alten Bestandes sei $(order,ist)$. Der neue Fahrzeugbestand sei
der Tupel $(order',ist')$.

$$
(order',ist') := \left \{\begin{array}{rl}
\mathbf{ab - ist > 0:} & (ist - ab + an, order)
\\
\mathbf{ansonsten:}    & (an, order + ist - ab)
\end{array}\right .$$
Am Anfang der Woche wird mit einem Startwert von $(0,0)$ begonnen, die Liste der
Fahrten nach Wochentagen wird dann von Montag bis Samstag mit der Funktion
\texttt{foldl} gefaltet.  Der Einfachheit halber wurde der Algorithmus zur
Bestimmung der An- und Abfahrten hart verdrahtet, mit geringerem Mehraufwand
lie"se sich der Algorithmus auch wesentlich generischer schreiben.

Die Ausgabe der Funktion sind drei Tupel, f"ur jede Station einen.  Es werden
allerdings nur die ersten Werte der Tupel gebraucht, der Rest wird am Ende
verworfen. Die Anzahl der nun wirklich ben"otigten Fahrzeuge ist die Summe jener
drei Werte.

\chapter{Teilaufgabe 2: Mit Leerfahrten}

\section{M"ogliche Verfahren}
Bevor ich mit der L"osung begann, habe ich über verschiedene Verfahren
res"umiert, wie man die Optimierungen m"oglichst effizient durchführen kann.
Folgende Ans"atze halte ich f"ur möglich:

\begin{itemize}
\item Genetischer Algorithmus
\item Brute Force
\item Darstellung aller m"oglicher Ver"anderungen als Suchbaum und dann Suche
nach bester Wahl mit geeignetem Verfahren.
\item M"oglicherweise komplexer, aber sytematischer Ansatz, der sehr schnell
la"uft.
\end{itemize}
Diese habe ich nach verschiedenen Kriterien (Laufzeitkomplexit"at, Komplexit"at
de Implementierung, \dots ) bewertet.

\section{Implementierter Algorithmus}
Ich habe mich am Ende f"ur die \emph{Brute Force}-L"osung entschieden, weil
diese vergleichsweise einfach zu implementieren ist.  Die hohe
Laufzeitkomplexit"at spielt aufgrund des geringen Ausma"ses des Problems nur
eine untergeordnete Rolle.

Das Verfahren besteht darin, zu untersuchen, wie sich die Hinzuf"ugung einzelner
Fahrten auf die Anzahl der n"otigen Fahrzeuge auswirkt.  Dazu wird von einer
Generatorfunktion eine Liste aller m"oglichen Auftragsb"ucher nach Hinzuf"ugung
\emph{einer} Fahrt generiert, die dann bewertet wird.  Alle Permutationen, die
mehr Fahrten ben"oetigen, werden nicht weiter verfolgt.  Bei Auftragsb"uchern,
die die gleiche Anzahl an Fahrzeugen ben"otigen, wird anhand der Parameter
entschieden.  Wenn die Anzahl der Fahrzeuge hingegen sinkt, so wird jenes Buch
in jedem Fall weiterverfolgt.

Dies wird Rekursiv durchgef"uhrt, eine Rekursionsebene ruft die Funktion zur
Optimierung der n"achsten Permutation nach o.\,g. Kriterien auf, und liefert das
beste Auftragsbuch zur"uck.

\subsection{Parameter}
Der Optimierungsalgorithmus wird "uber die Parameter $s$ und $i$ gesteuert. Der
Parameter $s$ steuert dabei, wie Tief bei gleichgro"ser Fahrzeugzahl abgestiegen
werden darf und der Parameter $i$ steuert die maximale Rekursionstiefe
ausgenommen jener Auftragsb"ucher, die die Anzahl der Fahrzeuge aufbessern.

\subsection{Einfluss der Parameter auf das Programm}
Ich habe durch diverse Tests herausgefunden, dass f"ur einige Auftragsb"ucher
die Parameter $s := 0,\ i := 0$ ausreichend sind, allerdings werden in diesem
Fall nicht immer gute Ergebnisse erzielt.  Falls man jedoch $s := 1,\ i := 0$
setzt, so braucht das Programm meist extrem lange zur Berechnung (Das ist auch
der Grund, warum ich den Parameter $i$ eingef"uhrt habe).

Eine gute Kombination scheint mir $s := 1,\ i := 3$, die Laufzeit hält sich so
meist in Grenzen und es werden hinreichend gute Optimierungen gefunden.  Unter
Verwendung dieser Parameter dauert die Optimierung der Test-Dateien zwischen 5-
und 20\ Sekunden.

\chapter{Verwendung des Programms}

\section{Parameter}
Das Programm kennt folgende Parameter:

\begin{verbatim}
Aufgabe3 [-1] [-2 [-s x] [-i x]] [-o Ausgabedatei] [Eingabedatei]
\end{verbatim}

Sollte der Parameter \texttt{-o} nicht gegeben sein, so auf die Standardausgabe
geschrieben, sollte keine Eingabedatei gegeben sein, so wird aus der
Standardingabe gelesen.

Die Parameter \texttt{-s} und \texttt{-i} steuern die zugeh"origen Parameter des
Optimierungsalgorithmus, Standardwerte sind $s := 1$ und $i := 3$. Die Parameter
\texttt{-1} und \texttt{-2} steuern, welche Teile der Aufgabe berechnet werden
sollen.

\section{Eingabeformat}
Das Programm akzeptiert Eingabedateien, die so formatiert sind wie die vom
Komitee bereitgestellten Testdateien.  Ung"ultige Dateien sorgen f"ur einen
Laufzeitfehler.

\section{Ausgabeformat}
Das Ausgabeformat gleicht dem Eingabeformat.  Die L"osung der 1.\ Teilaufgabe
wird folgenderma"sen ausgegeben, \texttt{\#\#\#} ist hierbei ein Platzhalter
f"ur die Anzahl der n"otigen Fahrzeuge:

\begin{verbatim}
#Teilaufgabe 1:
#Benötigte Fahrzeuge:
#Lager 1: ###
#Lager 2: ###
#Lager 3: ###
#Gesamt:  ###
\end{verbatim}

Die L"osung der 2.\ Teilaufgabe gibt zuerst den Kommentar \texttt{\#Teilaufgabe\ 
2} aus, anschlie"send folgt eine Darstellung der Touren inklusive Leerfahrten
im selben Format wie die Eingabe.  Zum Schluss wird die Anzahl der nun
ben"otigten Fahrzeuge im selben Format wie oben angegeben.

Sollte ein Aufgabenteil nicht berechnet werden, so wird die zugeh"orige L"osung
nicht in der Ausgabe enthalten sein, d.\,H. es wird nur der oben beschriebene
Teil ausgegeben.


\chapter{Quelltexte}

\section{Aufgabe3/Datatypes.hs}
\input{Datatypes.hs}

\section{Aufgabe3/Algorithmen.hs}
\input{Algorithmen.hs}

\section{Aufgabe3/IO.hs}
\input{IO.hs}

\section{Main.hs}
\input{Main.hs}

\end{document}
