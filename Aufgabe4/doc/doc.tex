\documentclass{scrreprt}

\usepackage[utf8]{inputenc}
\usepackage[T1]{fontenc}
\usepackage {ngerman}
\usepackage{fancyvrb}
\usepackage{xcolor}

\author {Team "`Herder"' -- Robert Clausecker}
\title {L"osung zur Aufgabe 4}
\subtitle {des 29. Bundeswettbewerbs Informatik}

%Definitionen für den Formatierer
\include{definitions}

\begin{document}
\maketitle

\tableofcontents \newpage

\chapter{Bedienung}

Das Programm l"asst sich in verschiedenen Modi starten.  Man kann sowohl einen
bestimmten Spielstand analysieren, als auch interaktiv gegen den Computer
spielen und sich eventuell helfen lassen.  Alle verf"ugbaren Modi werden im
folgenden erkl"art.

\section{Überblick}

Es folgt eine Auflistung der Betriebsmodi und weiterer Parameter.  Generell
werden Parameter, die keinen Sinn ergeben ignoriert.  Sollten wiederspr"uchliche
Parameter gegeben sein, so ist nur der jeweils letzte relevant.\footnote{Weitere
Parameter sind verf"ugbar, rufen Sie das Programm mit dem Schalter \texttt
{--help} f"ur weitere Informationen auf.}

\begin{description}
\item[-\,-\,help] Zeigt eine \emph{kurze Erl"auterung} der verf"ugbaren
  Parameter an.
\item[-\,-\,analyze] F"uhrt eine \emph{Analyse} eines Spielstandes aus.  Falls
  der Spielstand nicht per Parameter "ubergeben wird, so wird der Anfangszustand
  (also keine Karte umgedreht) verwendet.
\item[-\,-\,interactive] Startet den \emph{interaktiven Modus}.  Im interaktiven
  Modus kann man das Lernspiel gegen den Computer spielen.  Das Programm kann
  wahlweise auch Tipps geben und jederzeit Analysen durchf"uhren.
\item[-\,-\,auto] L"asst den Computer eine beliebige Anzahl an Spielen \emph
  {gegen sich selbst} spielen.  Die Ergebnisse werden gesammelt und ausgegeben.
\item[-\,s \textit{x}] Setzt den zu verwendenden Spielstand auf \textit{x}.
\item[-\,n \textit{x}] Setzt die Anzahl der Spiele, die der Computer gegen sich
  selbst spielt auf \textit{x}.
\item[-\,v] Gebe mehr Informationen aus.
\end{description}

\chapter{Quelltexte}

\section{Aufgabe4/Datentypen.hs}
\input{Datentypen.hs}

\section{Aufgabe4/Statistik.hs}
\input{Statistik.hs}

\section{Aufgabe4/IO.hs}
\input{IO.hs}

\section{Main.hs}
\input{Main.hs}

\end{document}
