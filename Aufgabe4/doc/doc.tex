\documentclass{scrreprt}

\usepackage[utf8]{inputenc}
\usepackage[T1]{fontenc}
\usepackage {ngerman}
\usepackage{fancyvrb}
\usepackage{xcolor}

\author {Team "`Herder"' -- Robert Clausecker}
\title {L"osung zur Aufgabe 4}
\subtitle {des 29. Bundeswettbewerbs Informatik}

%Definitionen für den Formatierer
\include{definitions}

\begin{document}
\maketitle

\tableofcontents \newpage


\chapter{Bedienung}

Das Programm l"asst sich in verschiedenen Modi starten.  Man kann sowohl einen
bestimmten Spielstand analysieren, als auch interaktiv gegen den Computer
spielen und sich eventuell helfen lassen.  Alle verf"ugbaren Modi werden im
folgenden erkl"art.

\section{Überblick}

Es folgt eine Auflistung der Betriebsmodi und weiterer Parameter.  Generell
werden Parameter, die keinen Sinn ergeben ignoriert.  Sollten wiederspr"uchliche
Parameter gegeben sein, so ist nur der jeweils letzte relevant.\footnote{Weitere
Parameter sind verf"ugbar, rufen Sie das Programm mit dem Schalter \texttt
{-\-help} f"ur weitere Informationen auf.}

\begin{description}
\item[-\,-\,help] Zeigt eine \emph{kurze Erl"auterung} der verf"ugbaren
  Parameter an.
\item[-\,-\,analyze] F"uhrt eine \emph{Analyse} eines Spielstandes aus.  Falls
  der Spielstand nicht per Parameter "ubergeben wird, so wird der Anfangszustand
  (also keine Karte umgedreht) verwendet.
\item[-\,-\,interactive] Startet den \emph{interaktiven Modus}.  Im interaktiven
  Modus kann man das Lernspiel gegen den Computer spielen.  Das Programm kann
  wahlweise auch Tipps geben und jederzeit Analysen durchf"uhren.
\item[-\,-\,auto] L"asst den Computer eine beliebige Anzahl an Spielen \emph
  {gegen sich selbst} spielen.  Die Ergebnisse werden gesammelt und ausgegeben.
\item[-\,s \textit{x}] Setzt den zu verwendenden Spielstand auf \textit{x}.  Der
  Spielstand kann wahlweise auch ohne vorangestelltes \texttt{-s} gesetzt
  werden.
\item[-\,n \textit{x}] Setzt die Anzahl der Spiele, die der Computer gegen sich
  selbst spielt auf \textit{x}.
\item[-\,v] Gebe mehr Informationen aus.  Mehrfache Angabe des Parameters
  \texttt{-v} gibt noch mehr Informationen aus.\footnote{Diese Funktion ist
  haupts"achlich zur Fehlersuche gedacht unf funktioniert nicht fehlerfrei.}
\end{description}
Die Argumente werden generell erst dann auf Fehler gepr"uft, wenn sie auch
wirklich verwendet werden.  Dies steht v"ollig im Einklang mit dem Prinzip der
Faulheit von Haskell.

\section{Automatischer Modus}
Dieser Modus wird durch die Angabe des Parameters \texttt{-\/-auto} aktiviert.
Der Computer spielt eine Folge von Spielen gegen sich selbst, die Ergebnisse
werden gesammelt und als "ubersichtliche Statistik ausgegeben.  Die gew"urfelten
Zahlen werden vom Zufallsgenerator des Systems bestimmt, somit ist die Ausgabe
des Programms bei jedem Lauf mit hoher Wahrscheinlichkeit unterschiedlich.

Das durchschnittliche Ergebnis nach vielen (c.\,a. 500) Spielen sollte der
Ausgabe der Analysefunktion "ahneln und kann als Kontrolle verwendet werden.

\subsection{Beispiel}
Im folgenden Beispiel wurden 250 Spiele gegen den Computer gespielt.  Um Papier
zu sparen, ist die Ausgabe gek"urzt, die vollst"andige Ausgabe ist im Ordner
"'Beispiele"' des Quelltextes zu finden.

\begin{verbatim}
$ ./Aufgabe4 -cn 250 -s 12345

Statistik aus 250 Spielen:
Durchschnittliches Ergebnis: 33.24799999999997
Verteilung der Ergebnisse:
	 2:     1 (0.40%)
	10:     2 (0.80%)
	12:     1 (0.40%)
    ... usw.
\end{verbatim}

\chapter{Quelltexte}

\section{Aufgabe4/Datentypen.hs}
\input{Datentypen.hs}

\section{Aufgabe4/Statistik.hs}
\input{Statistik.hs}

\section{Aufgabe4/IO.hs}
\input{IO.hs}

\section{Main.hs}
\input{Main.hs}

\end{document}
